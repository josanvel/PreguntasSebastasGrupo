\documentclass[12pt,oneside]{article}
\usepackage{geometry}                                % See geometry.pdf to learn the layout options. There are lots.
\geometry{a4paper}                                           % ... or a4paper or a5paper or ... 
\usepackage{graphicx}                                % Use pdf, png, jpg, or epsß with pdflatex; use eps in DVI mode
                                                                % TeX will automatically convert eps --> pdf in pdflatex                
\usepackage{amssymb}

\usepackage[spanish]{babel}                        % Permite que partes automáticas del documento aparezcan en castellano.
\usepackage[utf8]{inputenc}                        % Permite escribir tildes y otros caracteres directamente en el .tex
\usepackage[T1]{fontenc}                                % Asegura que el documento resultante use caracteres de una fuente apropiada.

\title{Set de Problemas - Sebesta}
\author{Lenguajes de Programación - ESPOL}

%\date{}                                                        % Activate to display a given date or no date

\begin{document}
\maketitle
\section{Introducción}
Las respuestas propuestas en este repositorio son producto del trabajo de los estudiantes de la materia ``Lenguajes de Programación'' de la ESPOL, correspondientes a las preguntas del libro de Robert Sebesta, Concepts of Programming Languages.
\section{Preguntas y Respuestas}
\subsection{Capítulo 5: Nombres, Enlaces y Alcances.}
\subsubsection{Pregunta 4:¿Qué es un alias?}
Cuando más de un nombre de una variable puede acceder a la misma  localización de memoria, las variables son llamadas ALIAS.
\subsubsection{Pregunta 5:¿A cual categoria de C++ siempre tiene aliasis en varias de referencia?}
Cuando un puntero en C++ se establece en el punto con la llamada de una variable, el puntero anula su referencia anterior,  esto se lo llama como variables aliases.
\subsubsection{Pregunta 6:¿Qué es l-value de una variable? ¿Qué es r-value?}
La dirección de una variable es conocida como l-value, y el valor de una variable es conocida como r-value.
\subsubsection{Pregunta 7:Defina biding y binding time}
BINDING,es la asociación entre un atributo y una entidad, o entre una variable y un tipo o valor o entre operadores y símbolos.
BINDING TIME, es el tiempo en el que se hace un binding.
Se puede dar en diferentes tiempos:
En el tiempo de la implementación del lenguaje.
En el tiempo de compilación.
En el tiempo de cargar.
En el tiempo de enlazado o el tiempo de ejecución.


\subsection{Capítulo 6: Tipos de Datos.}
\subsubsection{Pregunta 1:¿Qué es un descriptor?}
Es la colección de los atributos de unas variables.
En una implementación un descriptor es un área de la memoria que almacena los atributos de una variable, estos descriptores son construidos por el compilador, usualmente como la parte de un símbolo de una tabla y son usados durante la compilación.
\subsubsection{Pregunta 2:¿Cúales son las ventajas y desventajas de los tipos de datos decimal?}
Los tipos de DECIMAL tienen la ventaja de ser capaz de almacenar precisamente valores decimales, al menos aquellos dentro de un rango restringido, que no se puede hacer con un punto-flotante.
Las desventajas de los tipos decimales son que el rango  de valores está restringidas porque no se permiten exponentes y su representación en la memoria es ligeramente desperdiciada.
\subsubsection{Pregunta 7:¿En qué maneras la definición de usuario de tipo numerico de C\# es mas confiable que aquellas de C++?}
Los tipos numéricos de C\# son como aquellos de C++, excepto que ellos nunca serán coercionados a integer. Entonces, las operaciones de tipos numéricas son restringidas a que tenga ese sentido, al igual que los rangos de valores, serán restringidas a aquellas en particular de tipo numérico.
\subsection{Capítulo 7: Expresiones y Sentencias de Asignación.}
\subsubsection{Pregunta 1:Defina Operador de precedencia y operador de Asociatividad}
Operador de precedencia.- Dictan el orden de evaluación de los operadores.
Operador de asociatividad.- Cuando una expresión contiene 2 o más operando separados por un solo operador, cada uno de ellos y dichos operadores son del mismo nivel de precedencia, la respuesta de quien se evaluara primero, dependerán de las reglas de asociatividad del lenguaje
\subsubsection{Pregunta 2:¿Qué es un operador ternario?}
Un operador es ternario cuando tiene 3 operando.
\subsubsection{Pregunta 3:¿Qué es un operador Prefix?}
Prefix significa que los operadores preceden de sus operandos.
\subsubsection{Pregunta 4:¿Qué operador es usualmente de asociatividad derecha?}
En el lenguaje de Ruby y Fortram, el operador exponencial (\*\*) es la asociatividad usualmente derecha.
\subsubsection{Pregunta 5:¿Qué es un operador nonassociative?}
Supongamos que un programa solo evalúa:
A + B + C + D
Donde A y C son números positivos muy largos, B y D son números negativos muy largos, en la parte de evaluar A y B o C y D, esto no causaría desbordamiento, pero si se evalúa A y C o B y D, esto causaría desbordamiento. Porque las limitaciones de la aritmética computacional, en la adición es catastróficamente NO-ASOCIATIVA.
\subsubsection{Pregunta 6:¿Qué reglas de asociativdad son usadas por APL?}
En APL todos los operadores tienen el mismo nivel de precedencia. El orden de evaluación de los operadores en APL es determinado absolutamente por la regla de asociatividad del lenguaje, la regla es de derecha a izquierda, para todos los operadores.
\subsubsection{Pregunta 9:¿Qué es coerción?}
Es una conversión de tipo implícita que es iniciada en el compilador.
\subsection{Capítulo 8: Sentencias y Niveles de estructuras de control.}
\subsubsection{Pregunta 3:¿Qué es un block?}
Es una secuencia de código delimitado por cualquier llave y palabras finales reservadas. Los bloques pueden ser usados con métodos de escrituras especiales que crean muchos constructores útiles, incluyendo iteradores para estructuras de datos.
\subsubsection{Pregunta 4:¿Cúales son los errores de diseño para toda la selección y sentencias de control de iteración?}
Solo hay un problema de diseño de error para las declaraciones de selección y control de iteración:
¿En caso de la estructura de control tiene varias estructuras?
Todas las construcciones de selección e iteración controlan la ejecución de segmentos de códigos.
La pregunta es si la ejecución de dichos segmentos del código siempre comienza con la primera instrucción en el segmento. Ahora se cree generalmente que múltiples entradas añade un poco la flexibilidad de una instrucción de control, en la relación con la disminución de la legibilidad causada por el aumento de la complejidad.
\subsubsection{Pregunta 5:¿Qué son los errores de diseño para las estructuras de selección?}
¿Cuál es la forma y el tipo de la expresión que controla la selección? 
¿Cómo se especifican las cláusulas then y else? 
¿Cómo se debe especificar el significado de selectores anidados?

\subsection{Capítulo 9: Subprogramas.}
\subsubsection{Pregunta 1:¿Cúales son las 3 caracteristicas generales de los subprogramas?}
* Cada subprograma tiene un único punto de entrada.\\
* La unidad del programa de llamada se suspende durante la ejecución del llamado de un subprograma, en lo cual implica que hay solo un subprograma en ejecución en el momento dado.\\
* El control vuelve siempre a la persona que lo llama cunado la ejecución del programa termina.\\
\subsubsection{Pregunta 5:¿Qué lenguajes permiten tener una variable numerica como parametro?}
Python, Ruby, C++, Fortram 95, Ada, PHP

\end{document}