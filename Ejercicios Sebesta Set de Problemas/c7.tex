\subsubsection{Pregunta 1:Defina Operador de precedencia y operador de Asociatividad}
Operador de precedencia.- Dictan el orden de evaluación de los operadores.
Operador de asociatividad.- Cuando una expresión contiene 2 o más operando separados por un solo operador, cada uno de ellos y dichos operadores son del mismo nivel de precedencia, la respuesta de quien se evaluara primero, dependerán de las reglas de asociatividad del lenguaje
\subsubsection{Pregunta 2:¿Qué es un operador ternario?}
Un operador es ternario cuando tiene 3 operando.
\subsubsection{Pregunta 3:¿Qué es un operador Prefix?}
Prefix significa que los operadores preceden de sus operandos.
\subsubsection{Pregunta 4:¿Qué operador es usualmente de asociatividad derecha?}
En el lenguaje de Ruby y Fortram, el operador exponencial (\*\*) es la asociatividad usualmente derecha.
\subsubsection{Pregunta 5:¿Qué es un operador nonassociative?}
Supongamos que un programa solo evalúa:
A + B + C + D
Donde A y C son números positivos muy largos, B y D son números negativos muy largos, en la parte de evaluar A y B o C y D, esto no causaría desbordamiento, pero si se evalúa A y C o B y D, esto causaría desbordamiento. Porque las limitaciones de la aritmética computacional, en la adición es catastróficamente NO-ASOCIATIVA.
\subsubsection{Pregunta 6:¿Qué reglas de asociativdad son usadas por APL?}
En APL todos los operadores tienen el mismo nivel de precedencia. El orden de evaluación de los operadores en APL es determinado absolutamente por la regla de asociatividad del lenguaje, la regla es de derecha a izquierda, para todos los operadores.
\subsubsection{Pregunta 9:¿Qué es coerción?}
Es una conversión de tipo implícita que es iniciada en el compilador.